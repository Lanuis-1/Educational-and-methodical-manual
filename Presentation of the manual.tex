\documentclass{beamer}
\usepackage[utf8]{inputenc}
\usepackage[T2A]{fontenc}
\usepackage[english,russian]{babel}
\usepackage{amssymb,amsmath}
\usepackage{graphicx}
\usepackage{color}
\usepackage{listings}
\usepackage[format=plain,labelsep=period,justification=centerlast]{caption}
\usepackage{hyperref}
\usepackage{listings}
\usepackage{minted}
\usepackage{longtable}
\usepackage{ragged2e}
\usepackage{array}
\usepackage{graphicx}
\usepackage{epstopdf}
\usepackage{array}
\usepackage{longtable}
\usepackage{amsmath}
\usepackage{amssymb}
\usepackage{amsthm}
\usepackage{amsfonts}
\usepackage{braket}
\usepackage{gensymb}
\usepackage{physics}
\usepackage{listings}
\usepackage{graphicx}
\usepackage{xcolor}
\usepackage{movie15}
\setbeamertemplate{caption}[numbered]{}%
\usetheme{Madrid}
\usecolortheme{default}

\def\Year{\expandafter\YEAR\the\year}
\def\YEAR#1#2#3#4{#3#4}

\graphicspath{{figs/}}

\makeatletter
\defbeamertemplate*{footline}{my theme}{
	\leavevmode%
	\hbox{%
		\begin{beamercolorbox}[wd=.35\paperwidth,ht=2.25ex,dp=1ex,center]{author in head/foot}%
			\usebeamerfont{author and institute in head/foot}%
			\insertshortauthor~(\insertshortinstitute)
		\end{beamercolorbox}%
		\begin{beamercolorbox}[wd=.40\paperwidth,ht=2.25ex,dp=1ex,center]{title in head/foot}%
			\usebeamerfont{title in head/foot}
			\insertshorttitle
		\end{beamercolorbox}%
		\begin{beamercolorbox}[wd=.25\paperwidth,ht=2.25ex,dp=1ex,right]{date in head/foot}%
			\insertdate
			\hspace{0.5cm}
			\insertframenumber{} / \inserttotalframenumber\hspace*{2ex}
	\end{beamercolorbox}}%
}


\definecolor{color_1}{RGB}{210, 24, 26}
\definecolor{color_2}{RGB}{141, 29, 44}
\definecolor{color_3}{RGB}{94, 32, 40}
\setbeamercolor*{palette primary}{bg=color_1, fg=white}
\setbeamercolor*{palette secondary}{bg=color_2, fg=white}
\setbeamercolor*{palette tertiary}{bg=color_3, fg=white}
\setbeamercolor*{titlelike}{bg=color_1, fg = white}
\setbeamercolor*{title}{bg=color_1, fg = white}
\setbeamercolor*{caption name}{fg=color_1}
\usefonttheme{professionalfonts}


\title[Пособие для студентов]
{ Командная проектная работа
	на тему: «Пособие студентов для студентов «Запутались в кубитах?  Руководство по квантовым вычислениям»»}

\author[Журавлева А.Р, Коцур П.И, Бахтин А.Д.]
{Куратор проекта: преподаватель, к.ф.-м.н.
	Попова А.П.\\
	Команда:\\ Коцур П.И.,ФФБ-201-О-01\\
	Журавлева А.Р, гр ФФБ-201-О-01\\
	Бахтин А.Д, гр ФФБ-201-О-01}

\institute[ОмГУ]
{
	ОмГУ им. Ф.М.~Достоевского\\[5pt]
	
}

\date[Омск -- 20\Year]
{Омск -- 20\Year}

\begin{document}
	
	\frame{\titlepage}
	
	
	\begin{frame}\frametitle{Цель}
		\justifying \textbf {Наименование проекта}: Пособие студентов для студентов «Запутались в кубитах?  Руководство по квантовым вычислениям».\\
		
		\justifying \textbf{Время и место его осуществления}: 25.04.2025 – 20.06.2025, Омский государственный университет им. Ф.М. Достоевского.\\
		
		\justifying \textbf {Описание проблемы, решаемой в проекте}: Проблемой проекта является необходимость в составлении учебно-методического пособия с названием «Запутались в кубитах?  Руководство по квантовым вычислениям», разработанного для обеспечения базового понимания квантовых вычислений студентами младших курсов.\\
		
		\justifying  \textbf{Краткое содержание проекта}: Пособие с названием «Запутались в кубитах?  Руководство по квантовым вычислениям», которое поможет получить базовое понимание квантовых вычислений, путем анализа и систематизации информации. \\
	\end{frame}
 
 \begin{frame}\frametitle{Анализ заинтересованных сторон }
 		\setlength{\tabcolsep}{3pt}
 		\small
\begin{longtable}{|p{2.8cm}|p{1.5cm}|p{1.5cm}|p{2.6cm}|p{2.8cm}|}

	\hline
	\textbf{Стейкхолдер} & \textbf{Влияние (1-3)} & \textbf{Интерес (1-3)} & \textbf{Ожидания} & \textbf{Требования} \\ \hline
	Заказчик: 
	Научно-исследовательская лаборатория теоретической физики, прикладного моделирования и параллельных вычислений ОмГУ & 3 & 3 & 
	Составленное пособие «Запутались в кубитах?  Руководство по квантовым вычислениям»& 
	Проверенная систематизированная информация \\ \hline
	
	\textbf{Потребитель:} Студенты 3 курса ОмГУ & 1 & 3 & 
	Доступ к пособию «Запутались в кубитах?  Руководство по квантовым вычислениям»& 
	Проверенная систематизированная информация, к которой осуществляется быстрый и удобный доступ \\ \hline
\end{longtable}
 
\end{frame}
 
 \begin{frame}{Устав проекта (1/2)}
 	\footnotesize
 	\setlength{\tabcolsep}{4pt}
 	\renewcommand{\arraystretch}{1.2}
 	
 	\begin{longtable}{|p{3cm}|p{8cm}|}
 		\hline
 		\textbf{Цель проекта} & 
 		Составить учебное пособие «Запутались в кубитах? Руководство по квантовым вычислениям» к 20.06.2025 \\ \hline
 		
 		\textbf{Задачи проекта} & 
 		\begin{itemize}
 			\item Поиск и сбор информации о квантовых вычислениях
 			\item Работа с литературой
 			\item Анализ и систематизация информации
 			\item Составление пособия
 			\item Размещение на GitHub в открытом доступе
 		\end{itemize} \\ \hline
 		
 		\textbf{Результаты} & 
 		Готовое учебное пособие «Запутались в кубитах?» \\ \hline
 		
 		\textbf{Сроки} & 
 		28.04.2025 – 20.06.2025 (8 недель) \\ \hline
 	\end{longtable}
 \end{frame}
 
 \begin{frame}{Устав проекта (2/2)}
 	\footnotesize
 	\setlength{\tabcolsep}{4pt}
 	\renewcommand{\arraystretch}{1.2}
 	
 	\begin{longtable}{|p{3cm}|p{8cm}|}
 		\hline
 		\textbf{Риски проекта} & 
 		\begin{itemize}
 			\setlength{\itemsep}{0pt}
 			\item Отсутствие достаточных навыков и компетенций у студентов, реализующих проект;
 			\item Команда не успеет закончить проект в поставленный срок
 			\item Сокращение команды проекта
 			\item Технические проблемы
 		\end{itemize} \\ \hline
 		
 		\textbf{Команда} & 
 		\begin{itemize}
 			\setlength{\itemsep}{0pt}
 			\item Коцур П.И. — руководитель, технический специалист
 			\item Журавлева А.Р. — исполнитель
 			\item Бахтин А.Д. — аналитик
 		\end{itemize} \\ \hline
 		
 		\textbf{Критерии успеха} & 
 		\begin{itemize}
 			\setlength{\itemsep}{0pt}
 			\item Соблюдение сроков
 			\item Удовлетворение требований заказчика
 			\item Понятность изложения для студентов
 		\end{itemize} \\ \hline
 	\end{longtable}
 \end{frame}
 
 \begin{frame}\frametitle{Команда проекта }
 	\setlength{\tabcolsep}{3pt}
 	\begin{longtable}{|p{2.5cm}|p{2.5cm}|p{3cm}|p{3cm}|}
 		\hline
 		\textbf{Роль} & \textbf{ФИО} & \textbf{Функции} & \textbf{Результаты} \\ \hline
 		Руководитель & Коцур П. И. & Контроль проекта & Успешное завершение \\ \hline
 		Аналитик & Бахтин А. Д. & Анализ источников & Источники определены \\ \hline
 		Исполнитель & Журавлева А. Р. & Составление пособия & Пособие готово \\ \hline
 		Техспециалист & Коцур П. И. & Размещение на GitHub & Пособие в открытом доступе \\ \hline
 	\end{longtable} 
 \end{frame}
 
 \begin{frame}{Иерархическая структура работ (1/3)}
 	\scriptsize
 	\setlength{\tabcolsep}{4pt}
 	\renewcommand{\arraystretch}{1.3}
 	
 	\begin{longtable}{|c| p{4cm}|p{7cm}|}
 		\hline
 		\textbf{№} & \textbf{Работа (код)} & \textbf{Описание работы} \\ \hline
1.&\multicolumn{2}{|c|}{\textbf{ Подготовка к составлению учебно-методического пособия}} \\ \cline{2-3}
 		& 1.1. Определение предмета учебно-методического пособия & Определяем конкретную тематику для составления учебно-методического пособия \\ \cline{2-3}
 		& 1.2. Анализ имеющихся источников информации & Проводим исследование и анализ источников информации по теме \\ \cline{2-3}
 		& 1.3. Выбор источников информации & Ограничиваемся кругом полезной, интересующей нас литературы \\ \hline
 		2. & \multicolumn{2}{|c|}{\textbf{Поиск и сбор информации} } \\ \cline{2-3}
 		& 2.1. Распределение задач между исполнителями & Руководитель определяет обязанности каждого члена проекта в сборе информации \\ \cline{2-3}
 		& 2.2. Фактический сбор информации и начало составления & Собранная информация структурируется и используется во время составления \\ \hline
 	\end{longtable}
 \end{frame}
 

 \begin{frame}{Иерархическая структура работ (2/3)}
 	\scriptsize
 	\setlength{\tabcolsep}{4pt}
 	\renewcommand{\arraystretch}{1.3}
 	
 	\begin{longtable}{|c|p{4cm}|p{7cm}|}
 		\hline
 		\textbf{№} & \textbf{Работа (код)} & \textbf{Описание работы} \\ \hline
 		3. & \multicolumn{2}{|c|}{\textbf{Проверка и редактирование собранной информации}}  \\ \cline{2-3}
 		& 3.1. Определение эксперта по тематике & Поиск эксперта, готового анализировать собранную информацию \\ \cline{2-3}
 		& 3.2. Верификация полученной информации экспертом & Эксперт проверяет собранную информацию на предмет ошибок \\ \cline{2-3}
 		& 3.3. Редактирование внесенной информации экспертом & Эксперт помогает исправлять допущенные ошибки \\ \hline
 	\end{longtable}
 \end{frame}
 

 \begin{frame}{Иерархическая структура работ (3/3)}
 	\scriptsize
 	\setlength{\tabcolsep}{4pt}
 	\renewcommand{\arraystretch}{1.3}
 	
 	\begin{longtable}{|c|p{4cm}|p{7cm}|}
 		\hline
 		\textbf{№} & \textbf{Работа (код)} & \textbf{Описание работы} \\ \hline
 		4. &\multicolumn{2}{|c|}{ \textbf{Реализация проекта}}  \\ \cline{2-3}
 		& 4.1. Размещение на платформе GitHub & Составленное пособие размещается на платформе GitHub для удобного доступа \\ \cline{2-3}
 		& 4.2. Предоставление доступа студентам & Доступ к учебно-методическому пособию предоставляется студентам \\ \hline
 	\end{longtable}
 \end{frame}
 
 \begin{frame}{Вехи проекта}
 	\footnotesize
 	\setlength{\tabcolsep}{6pt}
 	\renewcommand{\arraystretch}{1.4}
 	
 	\begin{longtable}{|p{2cm}|p{5cm}|c|p{2cm}|}
 		\hline
 		\textbf{Веха} & \textbf{Результаты} & \textbf{Сроки} & \textbf{Ответственные} \\ \hline 		
 		1. Сбор информации & 
 		Изучена разная литература и выбрана необходимая информация для составления учебно-методического пособия. & 
 		30.04.2025 & 
 		Руководитель проекта \\ \hline
 		
 		2. Готовое учебно-методическое пособие & 
 		Составлено учебно-методическое пособие и доступ к нему предоставлен студентам & 
 		20.06.2025 & 
 		Руководитель проекта \\ \hline
 	\end{longtable}
 	
 \end{frame}
 	
\begin{frame}\frametitle{Матрица ответственности }
	\scriptsize
	\setlength{\tabcolsep}{3pt}
	\begin{longtable}{|m{3.5cm}|c|c|c|c|}
		\hline
		\textbf{Работа} & \textbf{Рук.} & \textbf{Анал.} & \textbf{Исп.} & \textbf{Техн.} \\ \hline
		Определение темы пособия & + & & & \\ \hline
		Выбор источников & + & + & & \\ \hline
		Анализ источников & + & + & & \\ \hline
		Распределение задач по разделам & + & & & \\ \hline
		Сбор информации и написание & & & + & \\ \hline
		Определение эксперта по тематике & + & + & & \\ \hline
		Проверка экспертомВерификация полученной информации экспертом & + & & & \\ \hline
		Редактирование внесенной информации экспертом & + & & + & \\ \hline
		Размещение на GitHub & & & & + \\ \hline
		Открытие доступа студентам & & & & + \\ \hline
	\end{longtable}
\end{frame}
\begin{frame}{Ресурсная матрица}
	\centering
	\footnotesize
	\setlength{\tabcolsep}{3pt}
	\renewcommand{\arraystretch}{1.2}
	
\begin{tabular}{|p{3cm}|p{3cm}|p{3.5cm}|p{1cm}|}
	\hline
	\textbf{Работы} & \textbf{Трудовые ресурсы} & \textbf{Материальные ресурсы} & \textbf{Иные} \\
	\hline
	Сбор и анализ данных & Персонал, студенты, куратор & Доступ к информационным ресурсам& --- \\
	\hline
	Редактирование и определение информации& Привлечение экспертов &Составленное пособие& --- \\
	\hline
	Предоставление доступа & Студенты, сотрудники &Составленное пособие & --- \\
	\hline
\end{tabular}
	
\end{frame}

\begin{frame}{Оценка рисков проекта}
\scriptsize
\setlength{\tabcolsep}{3pt}
\begin{tabular}{|c|l|c|c|p{2cm}|p{2.5cm}|}
	\hline
	\textbf{№} & \textbf{Риск} & \textbf{Вер/Вли} & \textbf{Оценка} & \textbf{Стратегия} & \textbf{Действия} \\
	\hline
	1 & Недостаток навыков & 0.5/0.5 & 0.25 & Передача & Доп. обучение \\
	\hline
	2 & Недостаток источников& 0.1/0.9 & 0.09 & Принятие & Консультация экспертов \\
	\hline
	3 & Несоблюдение сроков & 0.5/0.9 & 0.45 & Передача & Корректировка плана \\
	\hline
	4 & Сокращение команды & 0.1/0.9 & 0.09 & Принятие & Перераспределение задач \\
	\hline
	5 & Технические сбои & 0.1/0.5 & 0.05 & Минимизация & Резервное копирование \\
	\hline
\end{tabular}

\footnotesize
\textbf{Ответственные:} Руководитель проекта (1-4), Технический специалист (5)
\end{frame}
	\begin{frame}\frametitle{Глава 1: Основные положения}
		\justifying
		Данная глава посвещена основополагающим элементам теории квантовых вычислений. Представлены два параграфа:
		\begin{itemize}
			\item \justifying Кубит -- в этом параграфе дано определение основному понятию квантовых вычислений.
			\item \justifying Квантовое преобразование Фурье -- данный параграф посвящен одному из основных преобразований.
		\end{itemize}
		
	\end{frame}	
	\begin{frame}\frametitle{Глава 2: Ошибки}
		\justifying
		В квантовых вычислениях, подобно классическим, информация подвержена ошибкам. В этой главе  рассмотрены основные проблемы, связанные с возникновением ошибок в квантовых вычислениях, и почему методы, применяемые в классических вычислениях, здесь не всегда работают. Сама глава поделена на две подглавы, представим краткое описания.
		
	\end{frame}
	
	\begin{frame}\frametitle{Подглава 2.1: Основные проблемы}
		
		\begin{itemize}
			\item \justifying Декогерентизация -- 	Контакт квантового компьютера с окружающей средой (декогерентизация) является основной причиной ошибок, разрушающих квантовую информацию. Этому явлениею и посвящёен параграф.
			\item \justifying Неидеальность операций -- Даже идеальное изолирование от окружающей среды не гарантирует отсутствия ошибок. Про это явление написано в параграфе.
			\item \justifying Почему классические методы не работают.
			
		\end{itemize}
		
	\end{frame}
	
	\begin{frame}\frametitle{Подглава 2.2: Специфические трудности коррекции ошибок в квантовых вычислениях}
		\justifying
		Квантовая информация более хрупка, чем классическая, и требует специальных методов коррекции ошибок. именно этому  посвящена вторая подглава.:
		\begin{itemize}
			\item \justifying Фазовые ошибки.
			\item \justifying Малые ошибки.
			\item \justifying Измерение – причина возмущения.
			
		\end{itemize}
		
	\end{frame}
	
	\begin{frame}
		\frametitle{Глава 3: Квантовые вычисления и квантовые компьютеры}
		\justifying В данной главе рассматриваются квантовые вычисления, квантовые алгоритмы, устройство и применение квантовых компьютеров, а также представлено знакомство с синтаксисом языка для квантового компьютера OPENQASM 2.0 на примере генератора случайных чисел.
		
		Глава состоит из следующих параграфов:
		\begin{itemize}
			\item \justifying Квантовые вычисления. Однокубитовые и многокубитовые гейты. Состояние Белла
			\item \justifying Квантовый язык программирования OPENQASM 2.0. Генератор случайных чисел на квантовом компьютере
			\item \justifying Квантовый компьютер
		\end{itemize}
	\end{frame}
	
	\begin{frame}\frametitle{Глава 4: Квантовая информация}
		\justifying
		В этой главе происходит погружение в основы квантовой информации, изучение ключевых концепций и их удивительных приложений. Во многом затронутые темы носят фундаментальный характер. В главе представлены темы:
		\begin{itemize}
			\item \justifying Квантовая информация.
			\item \justifying Квантовая различимость.
			\item \justifying Квантовый параллелизм.
			\item \justifying Квантовая телепортация.
			\item \justifying Квантовая криптография.
			\item \justifying Невозможность клонирования.
		\end{itemize}
	\end{frame}
	
	
	\begin{frame}\frametitle{Глава 5: Реализация квантовых алгоритмов на языке квантового программирования }
		\justifying Глава "Реализация квантовых алгоритмов на языке квантового программирования"  содержит в себе примеры некоторых основных алгоритмов, основанных на квантовой версии преобразования Фурье, а также, само квантовое преобразование Фурье, пояснение кода, а также, объяснение программ, реализующих алгоритмы и задачи, которые могут помочь читающему лучше освоить материал.
	\end{frame}
	
	\begin{frame}
		\frametitle{Спасибо за внимание!}
		\begin{center}
			\Large Спасибо за внимание! \\
			\vspace{1cm}
			\href{https://github.com/Lanuis-1/Educational-and-methodical-manual.git}{ссылка на GitHub с материалами} \\
			\vspace{0.5cm}
			\includegraphics[width=3cm]{qrcode.png}
		\end{center}
	\end{frame}
\end{document}