\documentclass{beamer}
\usepackage[utf8]{inputenc}
\usepackage[T2A]{fontenc}
\usepackage[english,russian]{babel}
\usepackage{amssymb,amsmath}
\usepackage{graphicx}
\usepackage{color}
\usepackage{listings}
\usepackage[format=plain,labelsep=period,justification=centerlast]{caption}
\usepackage{hyperref}
\usepackage{listings}
\usepackage{minted}
\usepackage{ragged2e}
\usepackage{movie15}
\setbeamertemplate{caption}[numbered]{}%
\usetheme{Madrid}
\usecolortheme{default}

\def\Year{\expandafter\YEAR\the\year}
\def\YEAR#1#2#3#4{#3#4}

\graphicspath{{figs/}}

%% Выбор шрифтов %%
\usefonttheme[onlylarge]{structurebold}

% Привычный шрифт для математических формул
\usefonttheme[onlymath]{serif}

% Более крупный шрифт для подзаголовков титульного листа
\setbeamerfont{institute}{size=\normalsize}
%%%%%%%%%%%%%%%%%%%

% Если используется последовательное появление пунктов списков на
% слайде (не злоупотребляйте в слайдах для защиты дипломной работы),
% чтобы еще непоявившиеся пункты были все-таки немножко видны.
\setbeamercovered{transparent} \setlength\abovecaptionskip{-10pt}
\setlength\belowcaptionskip{-10pt}


\makeatletter
\defbeamertemplate*{footline}{my theme}{
	\leavevmode%
	\hbox{%
		\begin{beamercolorbox}[wd=.35\paperwidth,ht=2.25ex,dp=1ex,center]{author in head/foot}%
			\usebeamerfont{author and institute in head/foot}%
			\insertshortauthor~(\insertshortinstitute)
		\end{beamercolorbox}%
		\begin{beamercolorbox}[wd=.40\paperwidth,ht=2.25ex,dp=1ex,center]{title in head/foot}%
			\usebeamerfont{title in head/foot}
			\insertshorttitle
		\end{beamercolorbox}%
		\begin{beamercolorbox}[wd=.25\paperwidth,ht=2.25ex,dp=1ex,right]{date in head/foot}%
			\insertdate
			\hspace{0.5cm}
			\insertframenumber{} / \inserttotalframenumber\hspace*{2ex}
	\end{beamercolorbox}}%
}


\definecolor{color_1}{RGB}{210, 24, 26}
\definecolor{color_2}{RGB}{141, 29, 44}
\definecolor{color_3}{RGB}{94, 32, 40}
\setbeamercolor*{palette primary}{bg=color_1, fg=white}
\setbeamercolor*{palette secondary}{bg=color_2, fg=white}
\setbeamercolor*{palette tertiary}{bg=color_3, fg=white}
\setbeamercolor*{titlelike}{bg=color_1, fg = white}
\setbeamercolor*{title}{bg=color_1, fg = white}
\setbeamercolor*{item}{fg=color_1}
\setbeamercolor*{caption name}{fg=color_1}
\usefonttheme{professionalfonts}


\title[Пособие]
{ Командная проектная работа
	на тему: «Пособие студентов для студентов «Запутались в кубитах?  Руководство по квантовым вычислениям»»}

\author[Журавлева А.Р, Коцур П.И, Бахтин А.Д.]
{Куратор проекта:преподаватель, к.ф.-м.н.
	Попова А.П.\\
	Команда:\\ Коцур П.И.,ФФБ-201-О-01\\
	Журавлева А.Р, гр ФФБ-201-О-01\\
	Бахтин А.Д, гр ФФБ-201-О-01}

\institute[ОмГУ]
{
	ОмГУ им. Ф.М.~Достоевского\\[5pt]
	
}

\date[Омск -- 20\Year]
{Омск -- 20\Year}

\begin{document}
	
	\frame{\titlepage}
	
	
	\begin{frame}\frametitle{Цель}
		\justifying \textbf {Наименование проекта}: Пособие студентов для студентов «Запутались в кубитах?  Руководство по квантовым вычислениям».\\
		
		\justifying \textbf{Время и место его осуществления}: 28.04.2025 – 21.06.2025, Омский государственный университет им. Ф.М. Достоевского.\\
		
		\justifying \textbf {Описание проблемы, решаемой в проекте}: Проблемой проекта является необходимость в составлении учебно-методического пособия с названием «Запутались в кубитах?  Руководство по квантовым вычислениям», разработанного для обеспечения базового понимания квантовых вычислений студентами младших курсов.\\
		
		\justifying  \textbf{Краткое содержание проекта}: Пособие с названием «Запутались в кубитах?  Руководство по квантовым вычислениям», которое поможет получить базовое понимание квантовых вычислений, путем анализа и систематизации информации. \\
	\end{frame}
	
	
	\begin{frame}\frametitle{Глава 1: Основные положения}
		\justifying
		Данная глава посвещена основополагающим элементам теории квантовых вычислений. Представлены два параграфа:
		\begin{itemize}
			\item \justifying Кубит -- в этом параграфе дано определение основному понятию квантовых вычислений.
			\item \justifying Квантовое преобразование Фурье -- данный параграф посвящен одному из основных преобразований.
		\end{itemize}
		
	\end{frame}	
	\begin{frame}\frametitle{Глава 2: Ошибки}
		\justifying
		В квантовых вычислениях, подобно классическим, информация подвержена ошибкам. В этой главе  рассмотрены основные проблемы, связанные с возникновением ошибок в квантовых вычислениях, и почему методы, применяемые в классических вычислениях, здесь не всегда работают. Сама глава поделена на две подглавы, представим краткое описания.
		
	\end{frame}
	
	\begin{frame}\frametitle{Основные проблемы}
		
		\begin{itemize}
			\item \justifying Декогерентизация -- 	Контакт квантового компьютера с окружающей средой (декогерентизация) является основной причиной ошибок, разрушающих квантовую информацию. Этому явлениею и посвящёен параграф.
			\item \justifying Неидеальность операций -- Даже идеальное изолирование от окружающей среды не гарантирует отсутствия ошибок. Про это явление написано в параграфе.
			\item \justifying Почему классические методы не работают.
			
		\end{itemize}
		
	\end{frame}
	
	\begin{frame}\frametitle{Специфические трудности коррекции ошибок в квантовых вычислениях}
		\justifying
		Квантовая информация более хрупка, чем классическая, и требует специальных методов коррекции ошибок. именно этому  посвящена вторая подглава.:
		\begin{itemize}
			\item \justifying Фазовые ошибки.
			\item \justifying Малые ошибки.
			\item \justifying Измерение – причина возмущения.
			
		\end{itemize}
		
	\end{frame}
	
	\begin{frame}
		\frametitle{Глава 3: Квантовые вычисления и квантовые компьютеры}
		\justifying В данной главе рассматриваются квантовые вычисления, квантовые алгоритмы, устройство и применение квантовых компьютеров, а также представлено знакомство с синтаксисом языка для квантового компьютера OPENQASM 2.0 на примере генератора случайных чисел.
		
		Глава состоит из следующих параграфов:
		\begin{itemize}
			\item \justifying Квантовые вычисления. Однокубитовые и многокубитовые гейты. Состояние Белла
			\item \justifying Квантовый язык программирования OPENQASM 2.0. Генератор случайных чисел на квантовом компьютере
			\item \justifying Квантовый компьютер
		\end{itemize}
	\end{frame}
	
	\begin{frame}\frametitle{Глава 4: Квантовая информация}
		\justifying
		В этой главе происходит погружение в основы квантовой информации, изучение ключевых концепций и их удивительных приложений. Во многом затронутые темы носят фундаментальный характер. В главе представлены темы:
		\begin{itemize}
			\item \justifying Квантовая информация.
			\item \justifying Квантовая различимость.
			\item \justifying Квантовый параллелизм.
			\item \justifying Квантовая телепортация.
			\item \justifying Квантовая криптография.
			\item \justifying Невозможность клонирования.
		\end{itemize}
	\end{frame}
	
	
	\begin{frame}\frametitle{Глава 5: Реализация квантовых алгоритмов на языке квантового программирования }
		\justifying Глава "Реализация квантовых алгоритмов на языке квантового программирования" содержит в себе примеры некоторых основных алгоритмов, основанных на квантовой версии преобразования Фурье, а также, само квантовое преобразование Фурье, пояснение кода, а также, объяснение программ, реализующих алгоритмы и задачи, которые могут помочь читающему лучше освоить материал.
	\end{frame}
	
	\begin{frame}
		\frametitle{Спасибо за внимание!}
		\begin{center}
			\Large Спасибо за внимание! \\
			\vspace{1cm}
			\href{https://github.com/Lanuis-1/Educational-and-methodical-manual.git}{ссылка на GitHub с материалами} \\
			\vspace{0.5cm}
			\includegraphics[width=3cm]{qrcode.png}
		\end{center}
	\end{frame}
\end{document}